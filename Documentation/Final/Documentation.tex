\documentclass{article}
\usepackage[T1]{fontenc}
\usepackage[polish]{babel}
\usepackage[utf8]{inputenc}
\usepackage{biblatex}
\usepackage{titling,lipsum}
\usepackage{indentfirst}

\addbibresource{bibliography.bib}

\title{Grafy i sieci - wykrywanie społeczności\\dokumentacja końcowa}
\date{\today}
\author{Ireneusz Stanicki, Mateusz Śliwakowski,\\Bartłomiej Truszkowski, Przemysław Woźniakowski}

\begin{document}
	\begin{titlingpage}
		\maketitle
	\end{titlingpage}
	\pagenumbering{arabic}

\tableofcontents
\newpage

%-----------------------------------------------------%
%Wstęp
%-----------------------------------------------------%
\section{Wstęp}

%-----------------------------------------------------%
%Algorytmy
%-----------------------------------------------------%
\section{Algorytmy}

%-----------------------------------------------------%
%Girvan Newman
%-----------------------------------------------------%
\subsection{Algorytm Girvan-Newmana}
\subsubsection{Implementacja}

W przypadku tego algorytmu praca została rozpoczęta od podstaw, a więc tradycyjnej wersji algorytmu Girvan-Newman dla społeczności rozłącznych. Jak się okazało biblioteka igraph, z której korzystaliśmy, udostępnia wbudowaną funkcję do wyliczania parametru edge betweenness. Dzięki skorzystaniu z tej metody implementacja algorytmu sprowadziła się do krótkiej funkcji, która wyznacza ww. parametr, a następnie usuwa krawędź, dla której jest on największy. Algorytm ten nie podlegał dalszej analizie - wykorzystano go tylko do sprawdzenia poprawności danych, na których pracowaliśmy, a także określenia struktury wyniku użytej w całym algorytmu.\\

Aby uzyskać wynik w postaci społeczności nachodzących użyty został algorytm CONGA, do którego zastosowano własne modyfikacje. Także w tym wypadku pomocna okazała się biblioteka igraph, w której zaimplementowana jest funkcja do wyznaczania parametru vertex betweenness. Dzięki temu w każdym kroku uzyskano zbiory dwóch kluczowych parametrów dla przebiegu całego algorytmu.\\

Warto przybliżyć również strukturę wyniku, który był zwracany przez zmodyfikowany algorytm. Funkcja zwracała tablicę wartości modularnośći dla każdego z kroku algorytmu. Ponadto uzupełniana była także tablica tablic results, która zawierała wyjściowy dendrogram. Po wykonaniu kroku algorytmu do results doklejana była kolejna tablica, zawierającą krawędzie usunięte w danym kroku. W przypadku konstrukcji tej tablicy przyjęto następującą zasadę: gdy dochodziło do kopiowania wierzchołka pierwszą krawędź stanowiła pętle dla wierzchołka kopiowanego.\\

Chcąc odtworzyć rozwiązanie dla danego kroku z dendrogramu wyjściowego, konieczne było użycie tablicy Overlap, która również była uzupełniana w trakcie działania programu. Dla oryginalnych wierzchołków wartość w tej tablicy wynosiła -1, dla dodawanych wskazywała na numer wierzchołka, z którego został on skopiowany.\\

Wynik działania funkcji musiał jeszcze zostać zmodyfikowany w celu uzyskania najlepszego wyniku dla badanego grafu. Dzięki temu opierając się na współczynniku modularności określana była wartość docelowa dla danego grafu (więcej na temat w kolejnej sekcji). Następnie z wyjściowego dendrogramu graf był odtwarzany do kroku, któremu odpowiadał użyty parametr modularności. Kolejną część stanowiła modyfikacja takiego grafu do wizulizacji - konieczne było tutaj usunięcie kopii wierzchołków i ponowne złączenie ich z oryginalnymi, zachowując odpowiednie numery społeczności.

\subsubsection{Podjęte decyzje}
Pierwszą decyzję, jaką należało podjąć przy implementacji algorytmu, był wybór sposobu obliczania modularności. Miara ta jest jasno określona przez autorów oryginalnego algorytmu Girvan-Newman, lecz przy modyfikacjach dla społeczności nachodzących nie dysponujemy tego typu wzorem. Wobec tego jako miara modularności został wybrany tutaj wskaźnik użyty w przypadku algorytmu OCDLCE, który został odpowiednio dostosowany do algorytmu CONGA. Miara ta skupiona jest na jednej społeczności lokalnej, natomiast w opisywanym przypadku należało ją określić dla całego grafu. Wobec czego była ona wyliczana dla każdej uzyskanej społeczności, a następnie aby określić modularność dla całego grafu, posłożono się średnią arytmetyczną z uzyskanych wartości. Warto nadmienić, iż w postaci społeczności bez krawędzi wychodzących poza nią modularność z OCDLCE powinna zwrócić nieskończoność - w celach implementacji sprowadziliśmy ten parametr do wartości liczby wszystkich wierzchołków.\\

Kolejny ważny wybór dotyczący modularności opierał się na wyborze tej wartości, dla której uzyskany wynik jest najlepszy. Założenie początkowe miary z OCDLCE opierało się na tym, iż rezultat jest najlepszy wtedy, gdy największa jest wartość tego parametru. Niestety w takim przypadku wynik najlepszy zawsze osiągany byłby na początku, gdyż wszystkie krawędzie są wtedy wewnątrz społeczności (która de facto jest wtedy łączną składową grafu). Wobec tego także tutaj trzeba było zastosować pewne modyfikacje tej miary. Zdecydowano się na wybór maksimów lokalnych dla całej tablicy wartości modularności. Następnie z wybranych wartości dokonywano wyboru wartości wynikowej na podstawie określonego na wejściu kwartylu. Tą wartością można już było manipulować - im mniejsza jej wartość tym bardziej wynikowa społeczność była rozrzedzona. I analogicznie - chcąc otrzymać podział na większe grupy należało zwiększyć tę wartość.\\

Manipulację wielkością społeczności, jaką chcemy otrzymać, można było także uskutecznić przy pomocy preprocessingu. Polegał on na przejściu całego grafu przez algorytm, z tą różnicą, że jedyną zwracaną wartością była różnica pomiędzy wartościami edge i vertex betweenness. Bazując na wartościach z każdego kroku można było określić, jaki stopień rozdrobnienia społeczności chcemy uzyskać. W ten sposób do głównego wykonania programu można było przesłać parametr gap, który określał handicap, jaki dajemy usuwaniu krawędzi nad kopiowaniem wierzchołka. Im większą wartość dla tego parametru wybraliśmy z wyniku preprocessingu, tym społeczności wynikowe były liczniejsze (co jednak zależało także od wyboru końcowego, wspomnianego w poprzednim akapicie).\\

Przechodząc do esencji algorytmu należy wspomnieć o dokonanych własnych modyfikacjach. Początkowo założono, iż dla tego algorytmu modyfikacje będziemy stosować w procesie kopiowania wierzchołka. Zgodnie z tymi założeniami zaimplementowane zostały dwa rozwiązania. Pierwsze z nich stosuje podział wierzchołka wyłącznie na podstawie daty zawarcia znajomości. W ten sposób wyliczamy średnią z dat zawarcia znajomości dla każdej z incydentnych krawędzi danego wierzchołka, a następnie te z wartościami poniżej średniej 'przepinamy' do kopii wierzchołka, pozostawiając pozostałe z oryginałem.\\

Podobnie postępujemy w przypadku modyfikacji używającej liczby wspólnych znajomych. W tym wypadku zmieniony został jednak sposób wyliczania wartości, która stanowi odniesienie porównania przy dzieleniu. Dla danego wierzchołka wyliczana jest tutaj średnia ważona, gdzie wagę stanowi liczba wspólnych znajomych. O ile pierwszy sposób wyliczania powinien premiować prosty podział chronologiczny, tak w tym wypadku większy nacisk powinien zostać położony na bardziej zżyte ze sobą społeczności (osoby, które niekoniecznie poznały się w danym czasie, ale dłużej znajdowały się wspólnie w danej grupie).

%-----------------------------------------------------%
%LPA
%-----------------------------------------------------%
\subsection{Label Propagation Algorithm}
\subsubsection{Implementacja}
\subsubsection{Podjęte decyzje}

%-----------------------------------------------------%
%OCDLCE
%-----------------------------------------------------%
\subsection{Overlapping Community Detection by Local Community Expansion}
\subsubsection{Implementacja}
\subsubsection{Podjęte decyzje}

%-----------------------------------------------------%
%Louvain
%-----------------------------------------------------%
\subsection{Algorytm Louvain}
\subsubsection{Implementacja}
\subsubsection{Podjęte decyzje}

%-----------------------------------------------------%
%Dane testowe
%-----------------------------------------------------%
\section{Dane testowe}
Rozpoczynając pracę, zakładaliśmy, że opisane algorytmy będziemy badać na zbiorach danych udostępnianych przez firmę Facebook. Portal ten chyba najbardziej kojarzy się z ogromnym źródłem danych dotyczących społeczności. Niestety modyfikacje, które chcieliśmy zastosować, nie mogły zostać zrealizowane na datasetach ściągniętych z tego portalu. Żaden z szeroko dostępnych zbiorów nie zawierał informacji na temat liczby wspólnych znajomych, a także daty zawarcia znajomości. O ile zdobycie tej pierwszej wiadomośći jest bardzo proste przy użyciu nawet najbardziej naiwnego algorytmu brute force, tak uzyskanie dat można było przeprowadzić tylko w sposób losowy, co mogło by tylko zaburzyć osiągane wyniki. Wobec tego swe spojrzenie skierowaliśmy w stronę zbiorów, które na pierwszy rzut oka nie kojarzą się ze społecznościami tak dobrze jak Facebook.
%-----------------------------------------------------%
%NBA
%-----------------------------------------------------%
\subsection{NBA}
'I loved this game' - to slogan reklamowy towarzyszący najlepszej lidze koszykarskiej, który odnosi się także do jednego z autorów tej pracy. Wybór tego zbioru danych motywowany był prywatnymi zainteresowaniami autorów, które wiążą się także z ogólną wiedzą na temat samej ligi. Dzięki temu zyskaliśmy możliwość oceny osiąganych rezultatów 'ludzkim okiem', przez co można było je poprawiać i oceniać dzięki swego rodzaju eksperckiej wiedzy na temat NBA.

Swoją rolę odegrała tu także ogólna dostępność danych. Ich źródło stanowił portal basketball-reference.com, który stanowi internetową, koszykarską encyklopedię. Korzystając ze scrapera napisanego w języku JavaScript można było zautomatyzować proces wyszukiwania danych. Pomógł tutaj alfabetyczny spis koszykarzy - w ten sposób zdobyliśmy informację na temat każdego z nich, a także w łatwy sposób uzyskaliśmy liczbę wszystkich sezonów i klubów, w jakich w swojej karierze występował dany zawodnik. Następnie dane te zostały obrobione z wykorzystaniem SQL, w celu uzyskania relacji między zawodnikami, a także daty zawarcia znajomości. Liczba wspólnych znajomych została określona przy pomocy prostego algorytmu napisanego w języku C\#.

Graf, który uzyskaliśmy, składa się z 4800 wierzchołków. Każdy wierzchołek można przypisać zawodnikowi, który postawił swą stopę na parkietach NBA w ciągu ponad 70 lat istnienia ligi. Dwaj zawodnicy są ze sobą w relacji, jeśli kiedykolwiek zagrali ze sobą w jednym klubie. Data zawarcia znajomości to w tym wypadku data pierwszego meczu, w którym obaj zawodnicy mieli okazję ze sobą zagrać. Tworząc relacje w ten sposób uzyskaliśmy ponad 146 tysięcy krawędzi.
%-----------------------------------------------------%
%Filmweb
%-----------------------------------------------------%
\subsection{Filmweb}
Poszukując parametru daty zawarcia znajomości zwróciliśmy się także ku innej części branży rozrywkowej. Źrodło danych w tym wypadku stanowił portal filmweb.pl, który dysponuje ogromną bazą danych na temat aktorów i filmów. Skala tych danych jest tak duża, iż musieliśmy ograniczyć nasze prace tylko do polskich aktorów. W ten sposób uzyskaliśmy graf z 10 000 wierzchołków i 190031 krawędziami.

Sposób uzyskania danych był analogiczny do tego prezentowanego w przypadku grafu NBA. Skonstruowany w języku JavaScript scraper został wykorzystany do przejrzenia listy wszystkich polskich aktorów, by następnie dla każdego z nich znaleźć listę kolegów z branży, z którymi najczęśniej występował w jednym filmie. Ten zbiór uczyniliśmy znajomymi badanego aktora. Datę zawarcia znajomości w tym wypadku stanowiła data premiery pierwszego filmu/serialu, w którym dana dwójka ze sobą wystąpiła. Dodatkowo na prośbę prowadzącego zbiór danych uzupełniony został o liczbę filmów, w których dana dwójka wspólnie występowała. Informacja ta okazała się również przydatna w modyfikacji jednego z algorytmów.

Podsumowując, wierzchołki uzyskanego grafu stanowią polscy aktorzy, pobrani z ogólnodostępnej bazy filmweb. Dwaj aktorzy są ze sobą w relacji, jeśli jeden z nich znajduje się w zakładce 'najczęściej występował z ...' u drugiego. Motywację wyboru tego grafu jako testowego stanowiło ogólne zainteresowanie branżą filmową autorów pracy, a także ogólna dostępność danych. Sama etykietyzacja wierzchołków pozwoliła nam łatwiej oceniać poprawność działania algorytmów - zwłaszcza w początkowej fazie implementacji z wykorzystaniem tylko małych zbiorów danych.

%-----------------------------------------------------%
%Github
%-----------------------------------------------------%
\subsection{GitHub}

%-----------------------------------------------------%
%Wyniki
%-----------------------------------------------------%
\section{Wyniki}

%-----------------------------------------------------%
%Girvan Newman
%-----------------------------------------------------%
\subsection{Algorytm Girvan-Newmana}
\subsubsection{Analiza czasowa}
\subsubsection{Analiza jakościowa}

%-----------------------------------------------------%
%LPA
%-----------------------------------------------------%
\subsection{Label Propagation Algorithm}
\subsubsection{Analiza czasowa}
\subsubsection{Analiza jakościowa}

%-----------------------------------------------------%
%OCDLCE
%-----------------------------------------------------%
\subsection{Overlapping Community Detection by Local Community Expansion}
\subsubsection{Analiza czasowa}
\subsubsection{Analiza jakościowa}

%-----------------------------------------------------%
%Louvain
%-----------------------------------------------------%
\subsection{Algorytm Louvain}
\subsubsection{Analiza czasowa}
\subsubsection{Analiza jakościowa}

%-----------------------------------------------------%
%Porównanie
%-----------------------------------------------------%
\subsection{Porównanie}

%-----------------------------------------------------%
%Wnioski
%-----------------------------------------------------%
\subsection{Wnioski}

\newpage
\printbibliography

\end{document}


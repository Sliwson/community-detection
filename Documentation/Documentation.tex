\documentclass{article}
\usepackage[T1]{fontenc}
\usepackage[polish]{babel}
\usepackage[utf8]{inputenc}
\usepackage{biblatex}
\usepackage{titling,lipsum}

\addbibresource{bibliography.bib}

\title{Grafy i sieci - wykrywanie społeczności}
\date{08/04/2020}
\author{Ireneusz Stanicki, Mateusz Śliwakowski,\\Bartłomiej Truszkowski, Przemysław Woźniakowski}

\begin{document}
	\begin{titlingpage}
		\maketitle
	\end{titlingpage}
	\pagenumbering{arabic}

\tableofcontents
\newpage

\section{Wstęp}
\subsection{Przedstawienie problemu}
Badanie sieci społecznościowych jest podstawową dziedziną nauki o sieciach. Problem ten znacząco zyskał na znaczeniu, odkąd możliwy jest dostęp do olbrzymich zbiorów danych, dzięki działaniom takich firm jak Facebook, Google, czy Twitter. Sczególnie istotnym zagadnieniem, zarówno dla środowiska biznesowego jak i akademickiego, jest wyszukiwanie społeczności w grafach społecznościowych. Wykorzystuje się je w takich dziedzinach jak kryminalistyka, opieka zdrowotna, polityka, czy marketing\cite{paper1}.

\textbf{Grafem społecznościowym} nazywamy taki graf, który reprezentuje relacje między jednostkami. Najczęściej spotykanym przykładem jest struktura, gdzie wierzchołki identyfikują osoby, a krawędzie - relacje między danymi osobami. Tą relacją może być znajomość, lecz nic nie stoi na przeszkodzie aby definiować ją dowolnie - np. pytaniem 'Czy dane osoby wymieniły ze sobą wiadomość?'.

Podstawowa idea wykrywania społeczności opiera się na znajdowaniu grup wierzchołków, dla których liczba połączeń w obrębie społeczności jest znacząco wyższsza, niż liczba połączeń do wierzchołków spoza tej społeczności. Oczywiście definicja ta jest dosyć płynna - w tym problemie często algorytmy opiera się na pewnych heurystykach, które sprawdza się na rzeczywistych zbiorach danych. Weryfikacja takich rozwiązań nie jest prosta - często przeprowadzana jest ona manualnie, przez analizę wizualizacji. 

\subsection{Typowe podejścia}

\newpage
\section{Algorytm Girvana-Newmana}
\subsection{Opis algorytmu}
\subsection{Modyfikacje dla społeczności rozłącznych}
\subsection{Proponowane modyfikacje}
\subsection{Oczekiwania}

\newpage
\section{Label Propagation Algorithm}
\subsection{Opis algorytmu}
\subsection{Modyfikacje dla społeczności rozłącznych}
\subsection{Proponowane modyfikacje}
\subsection{Oczekiwania}

\newpage
\section{Overlapping Community Detection by Local Community Expansion}
\subsection{Opis algorytmu}
\subsection{Proponowane modyfikacje}
\subsection{Oczekiwania}

\newpage
\section{Algorytm Louveina}
\subsection{Opis algorytmu}
\subsection{Modyfikacje dla społeczności rozłącznych}
\subsection{Proponowane modyfikacje}
\subsection{Oczekiwania}

\newpage
\section{Porównanie}

\newpage
\section{Projekt z dziedziny zainteresowań osobistych}

\newpage
\section{Uwagi końcowe}
Sample z cytatem!\cite{sample}.

\printbibliography

\end{document}


\documentclass{article}
\usepackage[T1]{fontenc}
\usepackage[polish]{babel}
\usepackage[utf8]{inputenc}
\usepackage{biblatex}
\usepackage{titling,lipsum}

\addbibresource{bibliography.bib}

\title{Grafy i sieci - wykrywanie społeczności}
\date{08/04/2020}
\author{Ireneusz Stanicki, Mateusz Śliwakowski,\\Bartłomiej Truszkowski, Przemysław Woźniakowski}

\begin{document}
	\begin{titlingpage}
		\maketitle
	\end{titlingpage}
	\pagenumbering{arabic}

\tableofcontents
\newpage

\section{Wstęp}
\subsection{Przedstawienie problemu}
Badanie sieci społecznościowych jest podstawową dziedziną nauki o sieciach. Problem ten znacząco zyskał na znaczeniu, odkąd możliwy jest dostęp do olbrzymich zbiorów danych, dzięki działaniom takich firm jak Facebook, Google, czy Twitter. Sczególnie istotnym zagadnieniem, zarówno dla środowiska biznesowego jak i akademickiego, jest wyszukiwanie społeczności w grafach społecznościowych. Wykorzystuje się je w takich dziedzinach jak kryminalistyka, opieka zdrowotna, polityka, czy marketing\cite{paper1}.

\textbf{Grafem społecznościowym} nazywamy taki graf, który reprezentuje relacje między jednostkami. Najczęściej spotykanym przykładem jest struktura, gdzie wierzchołki identyfikują osoby, a krawędzie - relacje między danymi osobami. Tą relacją może być znajomość, lecz nic nie stoi na przeszkodzie aby definiować ją dowolnie - np. pytaniem 'Czy dane osoby wymieniły ze sobą wiadomość?'.

Podstawowa idea wykrywania społeczności opiera się na znajdowaniu grup wierzchołków, dla których liczba połączeń w obrębie społeczności jest znacząco wyższsza, niż liczba połączeń do wierzchołków spoza tej społeczności. Oczywiście definicja ta jest dosyć płynna - w tym problemie często algorytmy opiera się na pewnych heurystykach, które sprawdza się na rzeczywistych zbiorach danych. Weryfikacja takich rozwiązań nie jest prosta - często przeprowadzana jest ona manualnie, przez analizę wizualizacji. 

\subsection{Typowe podejścia}

\newpage
\section{Algorytm Girvana-Newmana}
\subsection{Opis algorytmu}
Metoda Girvan-Newmana jest metodą dzielącą - rozważanie problemu rozpoczyna się od całej sieci i z każdym krokiem zmniejsza ją poprzez usuwanie krawędzi. Każda z nich posiada specjalny parametr, który w literaturze funkcjonuje pod nazwą \textit{edge betweenness}. Określa on sumę najkrótszych ścieżek przechodzących przez daną krawędź. W przypadku tylko jednej takiej ścieżki parametr zwiększany jest o 1, natomiast gdy między wierzchołkami występuję więcej niż jedna najkrótsza ścieża, to każdej z tych ścieżek przypisywana jest równa waga, tak by suma wag dla wszystkich była równa 1.

Cecha ta zwraca od razu uwagę na fakt, który może stanowić główną wadę algorytmu - jego złożoność jest dość duża i może on mieć znaczne problemy wydajnościowe przy większych sieciach. Co interację, po usunięciu krawędzi, konieczne jest ponowne wyliczenie wartości parametru dla każdej z krawędzi. To prowadzi do złożoności czasowej na poziomie $O({m}^2n)$, gdzie m stanowi liczbę krawędzi, a n liczbę wierzchołków. Usprawnienia algorytmu można dokonać poprzez wyrzucenie w jednym kroku wszystkich krawędzi o najwyższej wadze. Co prawda nie zmieni to złożoności pesymistycznej algorytmu, jednak w większości przypadków znacznie go usprawni. Warto nadmienić, iż twórcy algorytmu nie sprecyzowali kolejności usuwania krawędzi o tej samej wadze, pozostawiając ten aspekt do rozwiązania przy implementacji.

Algorytm kończy swe działanie, gdy usunięta zostanie ostatnia krawędź. W czasie jego działania stworzony zostanie dendrogram, na podstawie którego będzie można odbudować podział społeczności na danym poziomie. Znalezienie optymalnej dekompozycji grafu stanowi w tym wypadku kluczowy problem. W tym przypadku analizowana metoda ukazuje swoje zalety. Co iteracje można tutaj wyliczyć współczynnik modularności i na samym końcu określić poziom, dla którego był on największy, a następnie odtworzyć dany stan sieci. W ten sposób metoda Girvan-Newmana jest w stanie wskazać nam najlepszy podział sieci na społeczności.

Motywację przy wyborze tego algorytmu stanowił fakt, iż jest to najbardziej znana metoda związana z wykrywaniem społeczności, o której wzmianki można odnaleźć w niemal każdej publikacji naukowej z tej dziedziny. Zagłebiając się w tematykę warto rozpocząć od niego swoją pracę. 
\subsection{Modyfikacje dla społeczności rozłącznych}
\subsubsection{CONGA}
W tradycyjnym wykorzystaniu algorytmu Girvan-Newmana każda kolejna iteracja prowadzi do podziału sieci na mniejsze części, które razem zawierają wszystkie wierzchołki grafu. Aby umożliwić wykrywanie społeczności nakładających się, musimy odnaleźć moment w którym dojdzie do podziału wierzchołka, tak aby znalazł się on w dwóch oddzielnych społecznościach. W algorytmie CONGA problem ten został rozwiązany poprzez wprowadzenie dodatkowego parametru dla każdego wierzchołka, w literaturze nazwanego \textit{vertex betweenness}.

Analogia do nazwy \textit{edge betweenness} wykorzystywanej w wersji podstatowej algorytmu nie jest przypadkowa. Parametr dla wierzchołka mówi nam bowiem o tym, ile najkrótszych ścieżek przechodzi przez ten wierzchołek. Dodatkowo dla każdego wierzchołka ${v}$ wartość tą można obliczyć bazując na wyliczonych wcześniej parametrów każdej z krawędzi zawierającej ${v}$. Zważywszy na złożoność czasową całego algorytmu modyfikacja ta nie jest znacząco kosztowna.

Z podejściem CONGA związane są trzy główne pytania na temat podziału wierzchołka:
\begin{enumerate}
\item Kiedy wykonać je zamiast zwykłego usunięcia krawędzi?
\item Który wierzchołek podzielić?
\item Jak go podzielić?
\end{enumerate} 

O ile na 2 pytanie odpowiedź stanowi parametr \textit{vertex betweenness}, tak pozostałe wymagają przyjęcia pewnych założeń. Określenie momentu podziału wierzchołka zamiast usunięcia krawędzi również wydaje się proste - klonowanie następuje wtedy, gdy największa wartość \textit{vertex betweenness} przekroczy największą wartość \textit{edge betweenness}. Odpowiedź na ostatnie pytanie stanowi już jednak o wiele poważniejszy problem i wymaga wprowadzenia dodatkowego parametru.

\textit{Pair betweeness} wierzchołka $v$ dla $(u,w)$, gdzie $u$ i $w$ są sąsiadami $v$ i $u \neq w$ jest to liczba najkrótszych ścieżek, które przechodzą zarówno przez krawędź $(u,v)$ jak i $(v,w)$. Obliczenie tego parametru dla każdej pary można przeprowadzić razem z wyliczaniem zwykłego \textit{edge betweenness}. Aby móc wykonać podział konieczne jest stworzenie dodatkowej konstrukcji opartej na ww. parametrze. Konstrukcja ta jest k-kliką, dla której każdy wierzchołek jest numerowany indeksem sąsiada $v$, a krawędzie $(u,w)$ są oznaczane wartością parametru \textit{pair betweenness} dla tej pary osiągniętym przy analizie z wierzchołka $v$.

Podział wierzchołka na podstawie parametru \textit{pair betweeness} jest zrealizowany metodą zachłanną. Warto zwrócić uwagę, iż modyfikacja ta wymagać będzie dodatkowej pamięci, a także zwiększy czas wykonania. Jest to też zdecydowanie najbardziej rozbudowany krok algorytmu CONGA.
\subsubsection{CONGO}
Optymalizacja algorytmu CONGA zrealizowana przez jego twórców. Skupia się ona na analizie lokalnej problemu, odrzucając wszystkie ścieżki o długości większej niż parametr $h$ (zazwyczaj dość niska wartość). Następnie problem aktualizacji wartości \textit{edge betweenness} odbywa się również wyłącznie lokalnie, przez co czas wykonania algorytmu znacznie spada. W naszych działaniach pominiemy jednak tę metodę, gdyż podając ją analizie uznaliśmy, iż ciężko znaleźć w niej miejsce do skutecznej modyfikacji.
\subsection{Proponowane modyfikacje}
\subsection{Oczekiwania}

\newpage
\section{Label Propagation Algorithm}
\subsection{Opis algorytmu}
\subsection{Modyfikacje dla społeczności rozłącznych}
\subsection{Proponowane modyfikacje}
\subsection{Oczekiwania}

\newpage
\section{Overlapping Community Detection by Local Community Expansion}
\subsection{Opis algorytmu}
\subsection{Proponowane modyfikacje}
\subsection{Oczekiwania}

\newpage
\section{Algorytm Louveina}
\subsection{Opis algorytmu}
\subsection{Modyfikacje dla społeczności rozłącznych}
\subsection{Proponowane modyfikacje}
\subsection{Oczekiwania}

\newpage
\section{Porównanie}

\newpage
\section{Projekt z dziedziny zainteresowań osobistych}

\newpage
\section{Uwagi końcowe}
Sample z cytatem!\cite{sample}.

\printbibliography

\end{document}


\documentclass{article}
\usepackage[T1]{fontenc}
\usepackage[polish]{babel}
\usepackage[utf8]{inputenc}
\usepackage{biblatex}
\usepackage{titling,lipsum}

\addbibresource{bibliography.bib}

\title{Grafy i sieci - wykrywanie społeczności}
\date{08/04/2020}
\author{Ireneusz Stanicki, Mateusz Śliwakowski,\\Bartłomiej Truszkowski, Przemysław Woźniakowski}

\begin{document}
	\begin{titlingpage}
		\maketitle
	\end{titlingpage}
	\pagenumbering{arabic}
	
\section{Wstęp}
\subsection{Przedstawienie problemu}
\subsection{Typowe podejścia}

\newpage
\section{Algorytm Girvana-Newmana}
\subsection{Opis algorytmu}
\subsection{Modyfikacje dla społeczności rozłącznych}
\subsection{Proponowane modyfikacje}
\subsection{Oczekiwania}

\newpage
\section{Label Propagation Algorithm}
\subsection{Opis algorytmu}
\subsection{Modyfikacje dla społeczności rozłącznych}
\subsection{Proponowane modyfikacje}
\subsection{Oczekiwania}

\newpage
\section{Overlapping Community Detection by Local Community Expansion}
\subsection{Opis algorytmu}
\subsection{Proponowane modyfikacje}
\subsection{Oczekiwania}

\newpage
\section{Algorytm Louveina}
\subsection{Opis algorytmu}
\subsection{Modyfikacje dla społeczności rozłącznych}
\subsection{Proponowane modyfikacje}
\subsection{Oczekiwania}

\newpage
\section{Porównanie}

\newpage
\section{Projekt z dziedziny zainteresowań osobistych}

\newpage
\section{Uwagi końcowe}
Sample z cytatem!\cite{sample}.

\printbibliography

\end{document}

